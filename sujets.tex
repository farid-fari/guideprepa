% Generated by mdtotex 0.5a
% see github.com/farid-fari/mdtotex

\documentclass{article}
\usepackage[utf8]{inputenc}
\usepackage[french]{babel}
\usepackage[T1]{fontenc}
\usepackage{geometry}
\geometry{left=30mm,right=30mm,top=25mm,bottom=50mm}
\usepackage{hyperref}
\usepackage{lmodern}
\usepackage{textcomp}
\setlength{\parindent}{0pt}

\begin{document}
\title{Liste de sujets recommandés}
\author{Farid Arthaud}
\maketitle
\tableofcontents

\vspace{1cm}

\textit{Note}: ce \texttt{pdf} contient des liens hypertexte, vérifiez que votre visionneuse \texttt{pdf} les détecte à l'aide de la table des contenus par exemple.

\textit{Note}: il s'agit d'un annexe au \texttt{guide.pdf} joint.

\textit{Version 0.2} - \textit{disponible sur \href{https://github.com/farid-fari/guideprepa}{github/farid-fari}}

\paragraph{Se procurer des sujets}\mbox{}\newline
Le site à consulter est \href{https://concours-maths-cpge.fr}{concours-maths-cpge}, qui recense tous les sujets depuis des décennies de tous les concours.
Il abrite en plus de nombreux corrigés, qui peuvent cependant parfois être de qualité peu désirable (attention aux ``c'est trivial'' non-triviaux).
Il vaut -- selon moi -- mieux se servir des corrigés seulement pour se décoincer lorsque l'idée manque, sans plus.

\part{Maths}
Les sujets de maths sont très variés, en difficulté comme en utilité.
Je note ici les sujets-clés que j'ai réalisés pendant l'année et durant la période de révisions.
Ce ne sont pas nécessairement les plus utiles, et la couverture du programme est large mais non approfondie.

\section{ENS}

\begin{tabular}{ |p{1cm} | p{2cm} | p{3cm} | p{8cm}| }
 Année  &  Sujet            &  Thème                                &  Notes \\
\hline\hline

 2004   &  Maths 2          &  Géométrie 3D                         &  Sujet \textbf{très} difficile et hors-programme en partie. L'originalité m'avait plu. \\\hline

 2010   &  Maths 1          &  Groupes                              &  Sujet \textit{assez} difficile mais faisable au moins en partie (\textit{début classique}). \\\hline

 2010   &  Maths 2          &  Groupes finis                        &  Passe aussi sur les espaces préhilbertiens. \\\hline

 2012   &  Second concours  &  Topologie                            &  Début absolument \textbf{incontournable}, troisième partie rude. \\\hline

 2014   &  Maths C          &  Espaces préhilbertiens, Intégration  &  Un début intéressant mais le sujet est \textbf{infaisable} sur la fin. \\\hline

\end{tabular}

\section{X}

\begin{tabular}{ |p{1cm} | p{2cm} | p{3cm} | p{8cm}| }
 Année  &  Sujet            &  Thème                                &  Notes \\
\hline\hline

 2010   &  Maths 1          &  Calcul différentiel                  &  Révision complète et accessible du calcul diff \\\hline

 2012   &  Maths B          &  Séries (toutes)                      &  Première et troisième parties classiques \\\hline

 2016   &  Maths B          &  Probabilités                         &  Sujet \textit{assez} difficile qui demande de l'astuce/intuition \\\hline

 2017   &  Maths B          &  Intégration                          &  Sujet \textbf{très} difficile, être rigoureux sur les détails \\\hline

\end{tabular}

\section{Centrale}

\begin{tabular}{ |p{1cm} | p{2cm} | p{3cm} | p{8cm}| }
 Année  &  Sujet            &  Thème                                &  Notes \\
\hline\hline

 2004   &  Maths 2          &  Analyse complexe                     &  Sujet très varié et intéressant, les \textit{similitudes} sont un point classique \\\hline

 2012   &  Maths 1 \textbf{PC}   &  Polynômes                            &  Sujet relativement \textit{facile}, premières questions classiques \\\hline

 2017   &  Maths 2          &  Probabilités                         &  Sujet très intéressant demandant beaucoup de rigeur (pièges omniprésents) \\\hline

\end{tabular}

\section{Mines}

\begin{tabular}{ |p{1cm} | p{2cm} | p{3cm} | p{8cm}| }
 Année  &  Sujet            &  Thème                                &  Notes \\
\hline\hline

 2017   &  Maths 2 \textbf{PC}   &  Analyse, Probabilités                &  Beaucoup de points classiques, et une conclusion satisfaisante (mais une difficulté faible) \\\hline

\end{tabular}

\part{Informatique}
\section{Commune}
Les sujets d'informatique commune balaient peu de terrain et se ressemblent souvent, tous concours confondus.
Il suffit de faire quelques sujets des années précentes pour s'en rendre compte.

On notera que CCP rassemble informatique option et commune, ou encore que les sujets de Centrale sont généralement de très bonne facture.

\section{Option}
\subsection{ENS/X}
Tous les sujets d'Info A sont généralement bons à partir de 2011 environ.
Je ne les conseille et comment pas individuellemt: certains sont plus ou moins faciles, mais l'ensemble est consistent et prédictible.
On notera cependant que les sujets très anciens (années 90) peuvent contenir des choses faisables et intéressantes en tant qu'entrainement -- mais plutôt pour les questions théoriques que de programmation (pas de OCaml explicitement, langage libre).

Ceci est moins vrai des sujets de Math-Info: le contraire l'est en fait.
Les sujets varient énormément entre les années, et il n'est au final pas très utile de les faire en très grandes quantité.
Je note simplement le sujet de 2018 (que j'ai fait au concours) que j'ai trouvé non seulement passionant, mais aussi bien construit et satisfaisant à faire (excepté pour quelques questions un peu calculatoires).
Attention, certains sujets sont très mauvais, tels que celui de 2016 qui contient des erreurs et des calculs fastidieux.

\subsection{Centrale, Mines, CCP}
Les sujets des autres concours suivent le même genre de motif que les sujets d'info A: ils ne sont pas crées égaux mais se ressemblent suffisament et sont suffisament consistents pour pouvoir être faits aléatoirement.

On notera encore tout de même certains intrus, comme le sujet de 2015 des Mines réputé très difficile.


\end{document}
