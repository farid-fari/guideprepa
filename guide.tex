% Generated by mdtotex 0.5a
% see github.com/farid-fari/mdtotex

\documentclass{article}
\usepackage{geometry}
\geometry{left=30mm,right=30mm,top=25mm,bottom=50mm}
\usepackage[utf8]{inputenc}
\usepackage[french]{babel}
\usepackage[T1]{fontenc}
\usepackage{hyperref}
\setlength{\parindent}{0pt}

\begin{document}
\title{Guide pour la MP*}
\author{Farid Arthaud}
\maketitle
\tableofcontents

\textit{Version 0.1 (\today)} - \textit{disponible sur \href{https://github.com/farid-fari/guideprepa}{github/farid-fari}}

\paragraph{Introduction: l'auto-censure}\mbox{}\newline
Un des pires vices en prépa est l'\textit{auto-censure}.
Il ne faut pas douter de ses capacités, ou s'imaginer qu'il existe une \textit{`caste'} d'élèves surdoués avec laquelle on ne peut se comparer.
Au-delà des frais d'inscription, poursuivre ses reves ne coute rien, et les différences de niveau en prépa sont souvent moins creusées qu'on ne voudrait l'imaginer.

L'important est donc de ne pas perdre le peu de confiance en soi restant de l'an précédent et de tout donner pour avoir \textbf{ses concours} sans se soucier de ceux des camarades.

\part{L'année scolaire}
\section{Maths}
\subsection{Cours}
Il peut etre utile de se procurer de certains ouvrages permettant de (mieux) comprendre le cours, ou bien de trouver des exercices supplémentaires.
Je suggère \textit{Analyse} et \textit{Algèbre} de \textit{Xavier Gourdon} (éditions \textit{Ellipse}) qui résument bien le cours et fournissent les exercices les plus classiques pour chaque chapitre.
Attention cependant à leur forte teneur en hors-programme.

\subsubsection{Apprentissage, Fiches}
Autant que l'an dernier, les fiches restent un outil puissant pour dominer le cours.
Personnellement, je fichais l'intégralité du cours sans noter les démonstrations, mais tous les formats raisonnables fonctionnent.

Certains chapitres de \textit{MP} sont constitués de nombreux théorèmes à ingurgiter (par exemple, \textit{Suites et séries de fonctions} ou \textit{Intégration}).
Connaitre les hypothèses de ces théorèmes et savoir rapidement les mettre en place en colle ou en DS est nécessaire -- bien que ce ne soit pas amusant du tout.

\subsubsection{Compréhension, Dessins}
Avoir une compréhension intuitive de ce qui se passe est important cette année, il faut donc faire particulièrement attention lorsque le prof gribouille un dessin sur un coin du tableau.
``Faire un dessin'' est d'ailleurs un conseil qui sera très pertinent plusieurs fois dans l'année, et il faut rapidement comprendre quelques astuces pour ceux-ci (vous verrez le sens de ces mots au fur et à mesure cette année).
\textbf{Attention, ces propositions sont \textit{extremement fausses} mais utiles pour approximer des dessins!}


\begin{itemize}
\item  Les fonctions sont \textit{monotones par morceaux}, souvent monotones
\item  Les ouverts sont \textit{connexes par arcs} et \textit{bornés}
\item  Les fonctions intégrables forment une \textit{``bosse'' autour de 0}
\item  Rien n'est `perpendiculaire' quand on n'a pas de structure euclidienne (produit scalaire)
\item  Si ca marche en dimension deux, ca marche en dimension $n$ -- faire les dessins dans le plan
\item  Les boules sont toujours des \textit{cercles}

\end{itemize}

En règle générale, aller au plus simple. En tirer une démonstration -- si elle fonctionne, rédiger; sinon complexifier le dessin jusqu'à avoir un cas suffisament général pour en tirer une démonstration.

Si on ne comprend pas un concept du cours et que les explications ne conviennent pas, se tourner au plus vite vers un ouvrage auxiliaire (voir ceux qui sont conseillés ci-dessus).

\subsection{TD}
Il n'est pas nécessaire de faire l'entièreté d'un TD pour maitriser un chapitre, cependant il faut etre capable de résoudre très rapidement les ``exercices-pièges'' (pièges classiques) et de connaitre les méthodes classiques.
Les TD sont très peu représentatifs des épreuves écrites qui sont plus longues et guidées, mais beaucoup plus des oraux.

Les concours visés doivent vous guider quant à la quantité du TD à faire, et sur la \textbf{nature} également.
Pour le concours \textit{ENS}, il faut savoir trouver une méthode de résolution de manière indépendante, et savoir faire des `sauts' entre les différents domaines pour trouver une méthode de résolution.
Pour les concours \textit{X}/\textit{Centrale}/\textit{Mines}, il faut savoir faire des calculs sinueux rapidement et \textit{sans erreurs}, et en plus trouver la méthode de calcul adaptée -- ce qui n'est pas toujours évident à l'\textit{X}.
Enfin, pour \textit{CCP}/\textit{E3A} il faut savoir maitriser les exercices classiques rapidement et se débrouiller devant des calculs (parfois numériques).

\subsection{DS}
Les DS deviendront au fur et à mesure de l'année de plus en plus des sujets de concours.
La notation de l'enseignant ne correspond pas nécessairement à celle des concours.

Une dichotomie difficile est le choix entre \textit{sauter} ou \textit{affronter} les questions sur lesquelles on coince.
C'est en fait une question de question de stratégie ou de gout (voire de style) qui n'admet pas de réponse finale.

\subsection{Colles}
Les colleurs de maths arrivent dans toutes les formes et coloris, et il faut savoir s'adapter -- c'est d'ailleurs aussi le cas des examinateurs de concours.
En général, ca sera le meme esprit que l'an dernier, mais il faut savoir garder son sang-froid devant un examinateur froid, distant voire insultant (on en a vu cette année dans toutes les banques).

\section{Physique}
Encore une fois, la connaissance du cours est primordiale.
Certains chapitres ont une thématique proche (par exemple, le bloc \textit{Electromagnétisme}) qui mènent à beaucoup de transversalité dans les sujets de concours.
Il faut donc etre capable de faire le lien entre les divers chapitres qualitativement (\textbf{analogies}) et quantitativement (résoudre des exercices faisant appel à plusieurs notions).

Il reste cependant parfois possible pendant les écrits de fuir un thème lorsque le sujet fait le tour du programme.

\subsection{TD}
Les TD seront aussi distants des sujets de concours qu'en maths, ils sont en fait une sorte de mélange oral-écrit (les exercices sont plus courts qu'un oral réel, et sont au final au plus proche des colles).
Ils sont cependant très utiles, puisqu'ils permettent de vérifier la compréhension de points précis du cours.
En général, ils sont terminables (une page recto-verso) et assurent une bonne note en DS s'ils sont maitrisés.

\section{Anglais}
L'important est de consommer un maximum de culture anglophone.
Cela vous permettera non seulement d'améliorer votre niveau \textit{technique} (prononciation, grammaire, vocabulaire...) et \textit{culturel} (connaissance des actualités, des évènements historiques marquants, ...).

Pour l'aspect culturel, l'objectif est de se former des repères précis pour les écrits autant que les oraux.

Je vous conseille les sources suivantes:


\begin{itemize}
\item  \textbf{Journaux} (\textit{NYT}, \textit{WaPo}, ... pour US et \textit{The Guardian}, \textit{BBC}, ... pour UK)
\item  \textbf{Radios} (\textit{NPR} pour US, \textit{BBC} pour UK)
\item  \textbf{Talk shows} américains (\textit{The Daily Show with Trevor Noah}, \textit{Late Night with Steven Colbert}, ...)
\item  Autres sources (séries/films, vidéos \textit{YouTube}, blogs, ...)

\end{itemize}

\section{Informatique}
Le niveau requis en informatique dépend énormément de vos objectifs.
Un élève préparant le concours \textit{info} des \textit{ENS} devra avoir une connaissance précise des divers algorithmes et complexités vues pendant l'année (ainsi que les implémentations) tandis qu'un élève en option SI visant le tétraconcours pourra exclusivement travailler à partir de sujets tombés les années précédentes.

Il n'est pas nécessaire de connaitre son cours par coeur en informatique, car les notions sont souvent floues et donc redéfinies dans les sujets.
Pour mieux comprendre cela, naviguer les divers sujets écrits d'informatique: les premières pages (re)définissent souvent les notions importantes du sujet.

\section{TIPE}
Il faut etre bien renseigné quant au coefficient du TIPE dans le concours vous intéressant: il peut etre très faible.
Notez qu'un contenu scientifique rigoureux n'est pas toujours nécessaire à l'obtention d'une bonne note, mais une présentation claire et pédagogue si.

Le TIPE est en fait un exercice de \textbf{communication scientifique}, et on vous reprochera beaucoup plus facilement une présentation difficile à suivre d'un sujet de très haute voltige qu'une présentation triviale mais claire d'un sujet banal.

\part{Les concours}
\section{Ecrits}

\subsection{Révisions}
Il n'y a pas de secret à la révision des écrits: il faut faire beaucoup de sujets.

\subsection{Organisation}
\subsection{Maths}
L'annexe ci-jointe (\texttt{sujets.pdf}) conseille quelques sujets explorés durant ma préparation que je trouve etre riches et intéressants.

\section{Oraux}
\subsection{Organisation}
\subsection{Maths}
Pour les concours \textit{X-ENS}, les éditions \textit{Cassini} proposent des exercices corrigés de très haute qualité.
Ces livres sont disponibles en pret à la bibliothèque universitaire (à SMH), et quelques \texttt{pdf} piratés trainent en ligne.
Attention à préparer des exercices vous concernant: préparer \textit{ENS} ne requiert pas de faire les exercices calculatoires proposés par \textit{Polytechnique}.

Tous concours confondus, l'important est de savoir dialoguer de manière constructive avec l'examinateur.
Il faut savoir montrer ce que l'on a compris (ou \textit{il n'y a pas de constats inutiles/stupides}) et ce qui nous fait coincer.
Attention, cela ne signifie aucunement qu'il faut parler à tout prix (ou \textit{il y a des constats inutiles/stupides}).
L'examinateur vous décoince lorsque vous coincez mais il ne fait pas le sujet à votre place -- voir par ailleurs la description des colleurs plus haut, qui reste valable pour les examinateurs.

\subsection{Anglais}
Les examinateurs, dépendant des concours, vont tester les choses suivantes:


\begin{itemize}
\item  \textbf{Compréhension} de la langue (tous)
\item  \textbf{Prononciation} en anglais (tous, surtout \textit{X-ENS})
\item  Connaissance de la \textbf{culture anglo-saxonne} (tous, surtout \textit{Mines})
\item  \textbf{Pertinence et rapidité} de la réflexion (tous)

\end{itemize}

Les examinateurs du concours \textit{X-ENS} (l'épreuve d'anglais est commune aux deux) vont vous poser des questions \textbf{très précises} concernant la vidéo: restitution des valeurs chiffrées dans la vidéo, d'une expression idiomatique particulière, etc ...
Ils aiment aussi poser des questions déstabilisantes, j'ai par exemple eu droit à ``What is the purpose of the video?''.

L'examinateur des \textit{Mines} que j'ai eu voulait sonder très précisément ma connaissance des actualités anglo-saxonnes (et la personne passant avant moi a eu droit au meme genre de questions).
Il faut donc etre capable de citer des exemples précis de tous les phénomènes évoqués.

A \textit{Centrale}, j'ai eu droit à des questions intéressantes et pertinentes autour du texte.
Une camarade a rapporté avoir été embétée pour s'etre trop éliognée du sujet dans son commentaire -- mais c'est un cas \textit{a priori} isolé.
Vous avez au début le choix entre deux textes, vous donnant de une flexibilité bienvenue.

\subsection{TIPE}
Il est important de s'entrainer à l'oral: présenter à diverses personnes et se préprarer à diverses questions.
Un jury peut etre intéressé par une petite partie de la présentation et poser de nombreuses questions dessus, il ne faut donc rien laisser de coté et etre pret à toutes les questions autour de son sujet si l'on veut réussir cette épreuve.
Les parties les plus simples peuvent mener à des questions plus pointues, comme des exemples.
La présentation doit contenir autant d'illustrations que possible, et vous devez etre pret à fournir et disséquer des exemples pour illustrer à peu près tout ce que vous présentez.

Ceci est d'autant plus vrai du concours \textit{ENS}, qui propose un format différent de TIPE (tandis que le concours de l'\textit{X} n'en propose pas).

\end{document}
