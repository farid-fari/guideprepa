% Generated by mdtotex 0.5a
% see github.com/farid-fari/mdtotex

\documentclass{article}
\usepackage[utf8]{inputenc}
\usepackage[french]{babel}
\usepackage[T1]{fontenc}
\usepackage{geometry}
\geometry{left=30mm,right=30mm,top=25mm,bottom=50mm}
\usepackage{hyperref}
\usepackage{lmodern}
\usepackage{textcomp}
\usepackage{titling}
\setlength{\droptitle}{-6em}
\setlength{\parindent}{0pt}

\begin{document}
\title{Guide pour la MP*}
\author{Farid Arthaud}
\maketitle
\tableofcontents

\textit{Note}: ce \texttt{pdf} contient des liens hypertexte, vérifiez que votre visionneuse \texttt{pdf} les détecte à l'aide de la table des contenus par exemple.

\textit{Version 0.2} - \textit{disponible sur \href{https://github.com/farid-fari/guideprepa}{github/farid-fari}}

\paragraph{L'auto-censure}\mbox{}\newline
Un des pires vices en prépa est l'\textit{auto-censure}.
Il ne faut pas douter de ses capacités, ou s'imaginer qu'il existe une \textit{`caste'} d'élèves surdoués avec laquelle on ne peut se comparer.
Au-delà des frais d'inscription, poursuivre ses rêves ne coute rien, et les différences de niveau en prépa sont souvent moins creusées qu'on ne voudrait l'imaginer.
Tout est faisable en un an, l'important est donc de rester confiant et de tout donner pour avoir ses concours.

\part{L'année scolaire}
\section{Maths}
\subsection{Cours}
Il peut être utile de se procurer de certains ouvrages permettant de (mieux) comprendre le cours, ou bien de trouver des exercices supplémentaires.
Je suggère \textit{Analyse} et \textit{Algèbre} de \textit{Xavier Gourdon} (éditions \textit{Ellipse}) qui résument bien le cours et fournissent les exercices les plus classiques pour chaque chapitre.
Attention cependant à leur forte teneur en hors-programme, même dans le cours.

\subsubsection{Apprentissage, Fiches}
Autant que l'an dernier, les fiches restent un outil puissant pour dominer le cours.
Personnellement, je fichais l'intégralité du cours sans noter les démonstrations, mais tous les formats raisonnables fonctionnent.

Certains chapitres de \textit{MP} sont constitués de nombreux théorèmes à ingurgiter (par exemple, \textit{Suites et séries de fonctions} ou \textit{Intégration}).
Connaître les hypothèses de ces théorèmes et savoir rapidement les mettre en place en colle ou en DS est nécessaire -- bien que ce ne soit pas amusant du tout.

\subsubsection{Compréhension, Dessins}
Avoir une compréhension intuitive de ce qui se passe est important cette année, il faut donc faire particulièrement attention lorsque le prof gribouille un dessin sur un coin du tableau.
``Faire un dessin'' est d'ailleurs un conseil qui sera très pertinent plusieurs fois dans l'année, et il faut rapidement comprendre quelques astuces pour ceux-ci (vous verrez le sens de ces mots au fur et à mesure cette année).
\textbf{Attention, ces propositions sont \textit{extrêmement fausses} mais utiles pour approximer des dessins!}


\begin{itemize}
\item  Les fonctions sont \textit{monotones par morceaux}, souvent monotones
\item  Les ouverts sont \textit{connexes par arcs} et \textit{bornés}
\item  Les fonctions intégrables forment une \textit{``bosse'' autour de 0}
\item  Rien n'est `perpendiculaire' quand on n'a pas de structure euclidienne (produit scalaire)
\item  Si ça marche en dimension deux, ça marche en dimension $n$ -- faire les dessins dans le plan
\item  Les boules sont toujours des \textit{cercles}

\end{itemize}

En règle générale, aller au plus simple.
En tirer une démonstration -- si elle fonctionne, rédiger; sinon complexifier le dessin jusqu'à avoir un cas suffisament général pour en tirer une démonstration.
Les dessins ne sont pas toujours la solution, mais l'intuition est un outil primordial.

Si on ne comprend pas un concept du cours et que les explications ne conviennent pas, se tourner au plus vite vers un ouvrage auxiliaire (voir ceux qui sont conseillés ci-dessus).

\subsection{TD}
Il n'est pas nécessaire de faire l'entièreté d'un TD pour maitriser un chapitre, cependant il faut être capable de résoudre très rapidement les ``exercices-pièges'' (pièges classiques) et de connaitre les méthodes classiques.
Les TD sont très peu représentatifs des épreuves écrites qui sont plus longues et guidées, mais beaucoup plus des oraux.

Les concours visés doivent vous guider quant à la quantité du TD à faire, et sur la \textbf{nature} des exercices à chercher également.
Pour le concours \textit{ENS}, il faut savoir trouver une méthode de résolution de manière indépendante, et savoir faire des `sauts' entre les différents chapitres pour trouver une méthode de résolution.
Il faut aussi savoir se remttre en question et \textbf{évaluer les pistes}: \textit{ai-je une chance de trouver la solution en procédant ainsi}?

Pour les concours \textit{X}/\textit{Centrale}/\textit{Mines}, il faut savoir faire des calculs sinueux rapidement et \textit{sans erreurs}, et en plus trouver la méthode de calcul adaptée -- ce qui n'est pas toujours évident à l'\textit{X}.
Enfin, pour \textit{CCP}/\textit{E3A} il faut savoir maitriser les exercices classiques rapidement et se débrouiller devant des calculs (parfois numériques).

\subsection{DS}
Les DS deviendront au fur et à mesure de l'année de plus en plus des sujets de concours.
Il ne faut en aucun cas négliger un sujet tombé en DS car il ``ne correspond pas aux concours visés''.
Un sujet, quelque soit le concours dont il est tiré, peut permettre de travailler une notion précise.
La notation de l'enseignant ne correspond pas nécessairement à celle des concours.

Une dichotomie difficile est le choix entre \textit{sauter} ou \textit{affronter} les questions sur lesquelles on coince.
C'est en fait une question de stratégie ou de gout (voire de style) qui n'admet pas de réponse finale.

\subsection{Colles}
Les colleurs de maths arrivent dans toutes les formes et coloris, et il faut savoir s'adapter -- c'est d'ailleurs aussi le cas des examinateurs de concours.
En général, ça sera le même esprit que l'an dernier, mais il faut savoir garder son sang-froid devant un examinateur froid, distant voire insultant (on en a vu cette année dans toutes les banques).

\section{Physique}
Encore une fois, la connaissance du cours est primordiale.
Certains chapitres ont une thématique proche (par exemple, le bloc \textit{Electromagnétisme}), ce qui mène à beaucoup de transversalité dans les sujets de concours.
Il faut donc être capable de faire le lien entre les divers chapitres \textit{qualitativement} (\textbf{analogies}) et \textit{quantitativement} (résoudre des exercices faisant appel à plusieurs notions).

Il reste cependant parfois possible pendant les écrits de fuir un thème lorsque le sujet fait le tour du programme.

Il est important de connaître quelques ordres de grandeur autant pour les oraux que les écrits (et surtout pour les \textit{Mines} et \textit{X/ENS}).

\subsection{TD}
Les TD seront aussi distants des sujets de concours qu'en maths, ils sont en fait une sorte de mélange oral-écrit -- les exercices sont plus courts qu'un oral réel, et sont au final au plus proche des colles.
Ils sont cependant très utiles, puisqu'ils permettent de vérifier la compréhension de points précis du cours.
En général, ils sont terminables (une page recto-verso) et assurent une bonne note en DS s'ils sont maitrisés.
Ils sont d'ailleurs souvent majoritairement traités en classe.

\subsection{DS}
Bien que ce soit contre-intuitif, il faut réellement savoir contourner les difficultés.
Parfois, le sujet fait appel à des notions floues (ce qui est beaucoup plus rare en \textit{MPSI}) et crée volontairement la confusion.
Devant une telle confusion, il faut soit \textit{construire} une réponse la plus satisfaisante possible - et en restant \textbf{cohérent} dans ses réponses -, soit \textit{sauter} cette question.

Il faut dans les deux cas savoir se séparer de son côté perfectionniste, pour donner une réponse un peu fragile mais la meilleure possible.

\section{Anglais}
L'important est de consommer un maximum de culture anglophone.
Cela vous permettera non seulement d'améliorer votre niveau \textit{technique} (prononciation, grammaire, vocabulaire...) mais aussi \textit{culturel} (connaissance des actualités, des évènements historiques marquants, ...).

Pour l'aspect culturel, l'objectif est de se former des repères précis pour les écrits autant que les oraux.

Je vous conseille les sources suivantes:


\begin{itemize}
\item  \textbf{Journaux} (\textit{NYT}, \textit{WaPo}, ... pour US et \textit{The Guardian}, \textit{BBC}, ... pour UK)
\item  \textbf{Radios} (\textit{NPR} pour US, \textit{BBC} pour UK)
\item  \textbf{Talk shows} américains (\textit{The Daily Show with Trevor Noah}, \textit{Late Night with Steven Colbert}, ...)
\item  \textbf{Autres} sources (séries/films, vidéos \textit{YouTube}, blogs, ...)

\end{itemize}

\section{Informatique}
Le niveau requis en informatique dépend énormément de vos objectifs.
Un élève préparant le concours \textit{info} des \textit{ENS} (voir section Concours) devra avoir une connaissance précise des divers algorithmes et complexités vues pendant l'année (ainsi que les implémentations) tandis qu'un élève en option SI visant le tétraconcours pourra exclusivement travailler à partir de sujets tombés les années précédentes.

Il n'est pas nécessaire de connaitre son cours par coeur en informatique, car les notions sont souvent floues et donc redéfinies dans les sujets.
Pour mieux comprendre cela, naviguer les divers sujets écrits d'informatique: les premières pages (re)définissent souvent les notions importantes du sujet.

\section{TIPE}
Il faut être bien renseigné quant au coefficient du TIPE dans le(s) concours vous intéressant: il peut être très faible.
Il ne faut donc pas dédier trop de ressources à cet exercice qui ne sera que superficiellement examiné, mais attention à ne pas le négliger et porter préjudice à son travail aux autres concours.
Notez qu'un contenu scientifique rigoureux et complet n'est (malheureusement) pas toujours nécessaire à l'obtention d'une bonne note, mais une présentation claire et pédagogue, si.

Le TIPE est en fait un exercice de \textbf{communication scientifique}, et on vous reprochera beaucoup plus facilement une présentation difficile à suivre d'un sujet de très haute voltige qu'une présentation triviale mais claire d'un sujet banal.

\textit{Remarque}: Si vous préparez les \textit{ENS}, il est conseillé de commencer à travailler sur le rapport très tôt.
Essayez de le rédiger en \LaTeX si possible.

\part{Les concours}
Aux mois de décembre et janvier, il est attendu de faire un choix d'écoles sur \textit{SCEI} (ce que je trouve être un peu tôt pour être certain).
Comme indiqué en introduction, il vaut \textit{a priori} mieux avoir trop de choix que pas assez.
Il reste toujours l'option de ne pas se présenter aux épreuves qui ne nous intéressent plus.
L'exercice de choix reste difficile puisqu'on peut se retrouver avec des sommes astronomiques à régler, il peut être sage de valider une variété de choix en difficulté, après consultation des \href{https://www.scei-concours.fr/statistiques.php}{statistiques \textit{SCEI}}.

\textit{Remarque}: valider l'inscription sur \textit{SCEI} ne la rend pas immuable, on peut toujours revenir et changer ses choix tant qu'on les revalide.
Le site est très peu clair sur ce point.

\section{Ecrits}
\textit{Remarque}: Respectez rigoureusement les limites de mots quand elles existent, c'est puni très sévèrement et il serait dommage qu'un sujet de langue détruise vos deux années de travail.

\subsection{Révisions}
Il n'y a pas de secret à la révision des écrits: il faut faire beaucoup de sujets.
Voici quelques points importants lors de la révision des écrits, dans aucun ordre particulier:


\begin{itemize}
\item  Savoir \textbf{hierarchiser} les sujets, il peut être plus sage de revoir une notion de maths mal maitrisée que de faire une synthèse d'anglais.
\item  Savoir \textbf{diversifier} les sujets, maitriser un éventail de chapitres (de préférence tous de manière égale)
\item  Toujours connaître le \textbf{format des épreuves} (temps, consignes, ...) et s'\textbf{habituer} à celles-ci
\item  Ne pas se \textbf{décaler}, garder un \textit{cycle horaire} proche de celui des écrits (\textit{8h-12h},\textit{14h-18h})
\item  Ne pas \textbf{rester seul\textperiodcentered e}, ne pas se \textbf{distraire}
\item  Savoir se \textbf{reposer}, ne pas se mettre en état de \textit{burn-out}

\end{itemize}

\subsection{Organisation}
L'organisation des écrits est généralement claire, les convocations arrivent quelques semaines en avance sur les sites respectifs des concours (ou par mail pour \textit{X}/\textit{ENS}).
Vous aurez (environ) une semaine par banque, et l'ordre en $2018$ était \textit{X}/\textit{ENS}, \textit{Centrale}, \textit{CCP} puis \textit{Mines} et \textit{E3A}.

Attention aux consignes particulières: pas de calculatrices aux \textit{Mines}, prénom sur la calculatrice à \textit{Centrale} (et calculatrice autorisée à toutes les épreuves scientifiques) ou encore papier limité à \textit{Centrale}.

\subsection{Maths, Informatique}
La plupart des sujets de maths et d'informatique depuis les années $80$ - tous concours confondus - peuvent être trouvés sur le site de l'\textit{UPS}, \href{https://concours-maths-cpge.fr/}{concours-maths-cpge.fr}.
De nombreux corrigés y sont aussi disponibles (attention cependant à leur qualité, la rédaction est parfois lapidaire).

L'annexe ci-jointe (\href{sujets.pdf}{\texttt{sujets.pdf}}) conseille quelques sujets explorés durant ma préparation que je trouve être riches et intéressants.

\section{Oraux}
\subsection{Organisation}
Chaque série d'oraux aura lieu durant une semaine, dans un ordre arbitraire qui vous sera révélé après les résultats d'admissibilité.
L'organisation est souvent plus branlante que celle des écrits, notamment pour le TIPE qui se passe au centre de Paris.

Les \textit{Mines} proposent trois sites de logement (\textit{ENSTA}, \textit{Mines ParisTech} et \textit{Ponts}).
L'organisation de l'\textit{X} est mêlée à celle de l'\textit{ENSTA} (certains oraux y ont lieu).
Les \textit{ENS} fournissent le strict minimum, et ont une organisation peu explicite.

De manière générale, vous serez logé sur place pendant vos oraux, mais les prix peuvent être élevés:


\begin{itemize}
\item  \textit{Centrale}: $\sim$220€ logements, $\sim$4€ par repas (2€ le petit-déjeuner)
\item  \textit{Mines} (site \textit{ENSTA}): $\sim$300€, repas compris
\item  \textit{CCP}: rien n'est fourni
\item  \textit{ENS}: $sim$200€ logement

\end{itemize}

\textit{Remarque}: Je conseille les abonnements \href{https://www.oui.sncf/bons-plans/tgvmax}{TGVmax} si vous comptez prendre le train.

\subsection{Sciences}
L'important est de savoir dialoguer de manière constructive avec l'examinateur.
Il faut savoir montrer ce que l'on a compris (ou \textit{il n'y a pas de constats inutiles/stupides}) et ce qui nous fait coincer.
Attention, cela ne signifie aucunement qu'il faut parler à tout prix (ou \textit{il y a des constats inutiles/stupides}).
L'examinateur doit vous décoincer lorsque vous coincez mais il ne fait pas le sujet à votre place -- voir par ailleurs la description des colleurs plus haut, qui reste valable pour les examinateurs.

Il faut encore une fois bien connaitre son cours: une bêtise ne coute rien (ou peu) si on sait répondre à la question de cours subséquente, mais est très couteuse si elle révèle une méconnaissance du cours.

\subsubsection{Maths}
Pour les concours \textit{X}/\textit{ENS} en maths, les éditions \textit{Cassini} proposent des exercices corrigés de très haute qualité.
Ces livres sont disponibles en prêt à la bibliothèque universitaire (à SMH), et quelques \texttt{pdf} piratés trainent en ligne.
Attention à préparer des exercices vous concernant: préparer les \textit{ENS} ne requiert pas de faire les exercices calculatoires proposés par \textit{Polytechnique}.

\subsubsection{Physique}
Les sujets de physique sont souvent trop longs pour la durée de l'oral.
Le format est relativement standard (hors \textit{X}/\textit{ENS}), avec une ou des questions de cours suivies au début et d'élargissement à la fin (énoncé d'une page environ composé de 5 à 10 questions).

Aux \textit{ENS}, on sera confronté à une question très courte et ouverte tandis qu'aux autres oraux on peut s'attendre à des exercices plus longs, en plusieurs questions.

\subsubsection{Informatique (\textit{ENS})}
Pour l'épreuve de TP, il faut être habitué aux distributions de type UNIX.
Je conseille de maitriser \texttt{vim} ou \texttt{emacs} puisque ce sont les seuls éditeurs respectables disponibles.
Je conseille de savoir coder en \texttt{ocaml} pour la vitesse supplémentaire (utiliser \texttt{ocamlopt}).
Attention aux questions théoriques ou de cours qui peuvent tomber pendant l'oral du TP.

Pour les épreuves d'informatique fondamentale, faire un maximum de sujets issus des rapports respectifs (\textit{LCR} et \textit{Ulm}).
Aucun corrigé n'existe \textit{a priori} mais les sujets sont faisables avec suffisament de temps.

\subsection{Anglais}
Les examinateurs, dépendant des concours, vont tester les choses suivantes:


\begin{itemize}
\item  \textbf{Compréhension} de la langue (tous)
\item  \textbf{Prononciation} en anglais (tous, surtout \textit{X}/\textit{ENS})
\item  Connaissance de la \textbf{culture anglo-saxonne} (tous, surtout \textit{Mines})
\item  \textbf{Pertinence et rapidité} de la réflexion (tous)

\end{itemize}

Les examinateurs du concours \textit{X}/\textit{ENS} (l'épreuve d'anglais est commune aux deux) vont vous poser des questions \textbf{très précises} concernant la vidéo: restitution des valeurs chiffrées dans la vidéo, d'une expression idiomatique particulière, etc ...
Ils aiment aussi poser des questions déstabilisantes, j'ai par exemple eu droit à ``What is the purpose of the video?''.

L'examinateur des \textit{Mines} que j'ai eu voulait sonder très précisément ma connaissance des actualités anglo-saxonnes (et la personne passant avant moi a eu droit au même genre de questions).
Il faut donc être capable de citer des exemples précis de tous les phénomènes évoqués.

A \textit{Centrale}, j'ai eu droit à des questions intéressantes et pertinentes autour du texte.
Une camarade a rapporté avoir été embêtée pour s'être trop éloignée du sujet dans son commentaire -- mais c'est un cas \textit{a priori} isolé.
Vous avez au début le choix entre deux textes, vous donnant une flexibilité bienvenue.

\subsection{TIPE}
Vous n'aurez pas d'autres oraux le jour du TIPE pour vous déplacer, et l'heure de convocation est en fait une heure d'avance sur l'heure de passage (dont 30 minutes d'avance sur l'heure d'appel).
Il a cependant lieu au centre de Paris (à l'\textit{IUT de Paris}, à environ 1 heure de Palaiseau en RER).
Ne soyez donc ni pressé, ni en avance.

Il est important de s'entrainer à l'oral: présenter à diverses personnes et se préprarer à diverses questions.
Un jury peut être intéressé par une petite partie de la présentation et poser de nombreuses questions dessus, il ne faut donc rien laisser de côté et être pret à toutes les questions autour de son sujet si l'on veut réussir cette épreuve.
Les parties les plus simples peuvent mener à des questions plus pointues, comme des exemples.
La présentation doit contenir autant d'illustrations que possible, et vous devez être prêt\textperiodcentered e à fournir et disséquer des exemples pour illustrer à peu près tout ce que vous présentez.

Ceci est d'autant plus vrai du concours \textit{ENS}, qui propose un format différent de TIPE (tandis que le concours de l'\textit{X} n'en propose pas).


\end{document}
