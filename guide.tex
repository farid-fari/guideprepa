% Generated by mdtotex 0.4
% see github/farid-fari/mdtotex

\documentclass{article}
\usepackage[utf8]{inputenc}
\usepackage{hyperref}
\setlength{\parindent}{0pt}

\begin{document}
\title{Guide pour la prépa}
\author{Farid Arthaud}
\maketitle
\tableofcontents

\textit{Version 0.1 (\today)}

\part{L'année scolaire}
\section{Maths}
\subsection{Cours}
\subsubsection{Fiches}

Autant que l'an dernier, les fiches restent un outil puissant pour dominer le cours.
Personnellement, je fichais l'intégralité du cours sans noter les démonstrations, mais tous les formats raisonnables fonctionnent.

Certains chapitres de \textit{MP} sont constitués de nombreux théorèmes à ingurgiter (par exemple, \textit{Suites et séries de fonctions} ou \textit{Intégration}).
Connaitre les hypothèses de ces théorèmes et savoir rapidement les mettre en place en colle ou en DS est nécessaire -- bien que ce ne soit pas amusant du tout.

\subsubsection{Dessins}

Avoir une compréhension intuitive de ce qui se passe est important cette année, il faut donc faire particulièrement attention lorsque le prof gribouille un dessin sur un coin du tableau.
``Faire un dessin'' est d'ailleurs un conseil qui sera très pertinent plusieurs fois dans l'année, et il faut rapidement comprendre quelques astuces pour ceux-ci (vous verrez le sens de ces mots au fur et à mesure cette année).
\textbf{Attention, ces propositions sont \textit{extremement fausses} mais utiles pour approximer des dessins!}


\begin{itemize}
\item  Les fonctions sont \textit{monotones par morceaux}, souvent monotones
\item  Les ouverts sont \textit{connexes par arcs} et \textit{bornés}
\item  Les fonctions intégrables forment une \textit{``bosse'' autour de 0}
\item  Rien n'est `perpendiculaire' quand on n'a pas de structure euclidienne (produit scalaire)
\item  Si ca marche en dimension deux, ca marche en dimension $n$ -- faire les dessins dans le plan
\item  Les boules sont des \textit{cercles} pour toutes les normes

\end{itemize}

En règle générale, aller au plus simple. En tirer une démonstration -- si elle fonctionne, rédiger; sinon complexifier le dessin jusqu'à avoir un cas suffisament général pour en tirer une démonstration.

\subsection{TD}

Il n'est pas nécessaire de faire l'entièreté d'un TD pour maitriser un chapitre, cependant il faut etre capable de résoudre très rapidement les ``exercices-pièges'' (pièges classiques) et de connaitre les méthodes classiques.
Les concours visés doivent vous guider quant à la quantité du TD à faire,

\subsection{DS}



\section{Physique}
\section{Francais}

Comme on va beaucoup le répéter, il faut avant out \textit{connaitre ses oeuvres}.

\section{Anglais}

L'important est de consommer un maximum de culture anglophone.
Cela vous permettera non seulement d'améliorer votre niveau \textit{technique} (prononciation, grammaire, vocabulaire...) et \textit{culturel} (connaissance des actualités, des évènements historiques marquants, ...).

Pour l'aspect culturel, l'objectif est de se former des repères précis pour les écrits autant que les oraux.

Je vous conseille les sources suivantes:


\begin{itemize}
\item  \textbf{Journaux} (\textit{NYT}, \textit{WaPo}, ... pour US et \textit{The Guardian}, \textit{BBC}, ... pour UK)
\item  \textbf{Radios} (\textit{NPR} pour US, \textit{BBC} pour UK)
\item  \textbf{Talk shows} américains (\textit{The Daily Show with Trevor Noah}, \textit{Late Night with Steven Colbert}, ...)
\item  Autres sources (séries/films, vidéos \textit{YouTube}, blogs, ...)

\end{itemize}

\section{Informatique}

Le niveau requis en informatique dépend énormément de vos objectifs.
Un élève préparant le concours \textit{info} des ENS devra avoir une connaissance précise des divers algorithmes et complexités vues pendant l'année (ainsi que les implémentations) tandis qu'un élève en option SI visant le tétraconcours pourra exclusivement travailler à partir de sujets tombés les années précédentes.

Il n'est pas nécessaire de connaitre son cours par coeur en informatique, car les notions sont souvent floues et donc redéfinies dans les sujets.
Par exemple, un arbre a diverses définitions possibles (binaire, général; feuilles ou arbres vides; ...).
Pour mieux comprendre cela, naviguer les divers sujets écrits d'informatique \textit{X-ENS}: les premières pages (re)définissent les notions importantes du sujet.

\section{TIPE}
\part{Les concours}
\section{Ecrits}

Il n'y a pas de secret à la révision des écrits: il faut faire beaucoup de sujets.

\section{Oraux}

Les examinateurs, dépendant des concours, vont tester les choses suivantes:


\begin{itemize}
\item  \textbf{Compréhension} de la langue (tous)
\item  \textbf{Prononciation} en anglais (tous, surtout \textit{X-ENS})
\item  Connaissance de la \textbf{culture anglaise} (tous, surtout \textit{Mines})
\item  \textbf{Pertinence et rapidité} de la réflexion (tous)

\end{itemize}

\subsection{ENS}
\subsection{X}
\subsection{Centrale}
\subsection{Mines}
\subsection{CCP}

\end{document}
